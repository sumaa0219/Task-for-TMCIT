\documentclass[11pt,dvipdfmx]{jarticle}

\usepackage{eee}
\usepackage{subfig}
\usepackage{here}

\renewcommand{\labelenumi}{\alph{enumi}}
\renewcommand{\labelenumii}{\roman{enumii}}

\newcounter{mycounter} % カウンタの宣言
\setcounter{mycounter}{0} % カウンタの初期化
\newcommand{\useMycounter}[1][]{\refstepcounter{mycounter}{#1}プログラム{\themycounter}: }
\begin{document}

% トップページを書く
\begin{jikkenTitle}
	\gakunen{3} % 学年を記述。この行で全体の枠を表示
	\numTitle{1}{電気電子計測} % 実験番号、タイトルを記述
	\subTitle{} % サブタイトルがあれば記述
	\jikkenbi{令和4年5月12日(木)} % 実験日を記述
	\jikkenbiII{令和4年5月19日(木)} % 実験日を記述(二日目がある場合。ない場合はこの行をコメントアウト)
	\kyoudou{なし} % 共同実験者名を記述
	\kyoudouII{} % その他の共同実験者名を記述
	\yoteibi{5/19}% 予定日を記述
	\yoteibiII{6/2}% 予定日2を記述
	\yoteibiIII{}% 予定日3を記述
	\hanNumberName{1}{3317}{杉山 滉太} % 班番号・学生番号・氏名を記述。この行でタイトルページの描画を終了
\end{jikkenTitle}

\section{目的}
本実験では
\begin{itemize}
	\item LabVIEWとMyRIOを使用して、素子の電圧電流特性について自動計測の方法を習得する。
	\item 測定データから近似直線式の傾き、切片を求める計算方法を習得する。
	\item 電圧電流特性から抵抗値を求める方法について習得する。
\end{itemize}
ことを目的とする。

\section{原理}
\subsection{LabVIEW}
LabVIEWは、各種計測器やmyRIOなどを用いて自動計測や制御を実装するためのグラフィカルユーザーインターフェイスのプログラミング言語である。
主な特徴は、ビルトインされた仮想計測器(以下VI)で、オシロスコープやマルチメーターなどの計測器と似た外観や機能をコンピューター上へ作成するというものである。
VIは、フロントパネル、ブロックダイアグラム、アイコン-コネクタという3つ主要素から構成される。
プログラミングは、ブロックダイアグラム上にアイコンを配置し、各アイコン間のコネクタをつなぐ形で行う。

\subsection{myRIO}
myRIOは、デュアルコアのARM Cortex-A9 リアルプロセッサとカスタマイズ可能なXilinx FPGA・アナログプロセッサの駆動するプログラミング言語には、LabVIEWを用いる。
LabVIEWとmyRIOを用いることにより、制御、ロボット、メカトロニクス、組込などを容易に実現することができる。

\subsection{myRIO ブレッドボードアクセサリ}
myRIOの拡張ポートに接続可能なブレッドボードアクセサリである。
myRIOの5V、3.3V、GND端子及びAnalog I/O、Digital I/Oの端子が、ブレッドボード上に結線した回路とヘッダにマッピングされている。
そのため、ブレッドボード上に結線した回路とヘッダとをジャンパ戦で結線することにより、回路への入出力制御および計測がmyRIOを用いて容易に実行することができる。

\subsection{真値と誤差及び相対誤差(誤差率)}
\subsubsection{真値}
真値とは、測定量(測定値ではない)が単位の何倍であるのかを示している値である。
真値は必ず存在すると仮定しても我々は真値そのものは知ることができず、ただその存在する範囲を推定することが出来るだけである。

\subsubsection{誤差及び相対誤差}
誤差は\weq{gosa}で定義される。
\begin{eqnarray}
	誤差 = 測定値 − 真値 
	\label{eq:gosa}
\end{eqnarray}
また、相対誤差とは真値に対する誤差の比である。
但し真値は分からないので、通常は\weq{soutaigosa}のように誤差が小さいとして真値の代わりに測定値で割る。
\begin{eqnarray}
	相対誤差 = \frac{誤差}{真値} \simeq \frac{誤差}{測定値}
	\label{eq:soutaigosa}
\end{eqnarray}

\subsection{統計処理(正規分布・平均値・標準偏差)}
\subsubsection{正規分布}
左右対称の釣鐘型に値が分布しているのを正規分布といい、山の頂点に平均値がくる。

\subsubsection{平均値}
平均値とは$N$個全てのデータの総和を$N$個で割って得られる値で、\weq{heikinchi}で表すことができる。\cite{a}
\begin{equation}
	\bar{y} = \frac{1}{N}\sum^N_{i = 1}y_i
	\label{eq:heikinchi}
\end{equation}		

\subsubsection{標準偏差}
標準偏差とは平均値を基準に各測定量がどれほどのばらついているかを定量的に表す値で、\weq{hyoujunhensa}で表すことができる。\cite{a}
\begin{equation}
	\sigma = \sqrt{\frac{1}{N - 1}\sum^N_{i = 1}(y_i - \bar{y})^2}
	\label{eq:hyoujunhensa}
\end{equation}	

\subsection{近似直線(最小二乗法)}
%。
2つの測定データ$y, x$間に一次方程式の関係があるとし、
\begin{equation}
	y = ax + b
	\label{eq:aiu}
\end{equation}
の傾き$a$、切片$b$を測定データから尤もらしい値にすることを考える。
その際に、
\begin{eqnarray}
	E &=& \sum\limits_{i=1}^{N} \varepsilon^2_i \nonumber\\
	&=& \sum\limits_{i=1}^{N} \bigl( y_i - f(x_i)\bigr)^2\nonumber\\\
	&=& \sum\limits_{i=1}^{N} \bigl( y_i - (ax_i+b)\bigr)^2
	\label{eq:error}
\end{eqnarray}
を最小にする$a$、$b$を求める。
これを最小二乗法といい、誤差を伴う測定値の処理においてその誤差の二乗の和を最小にすることで、最も確からしい関係式を求める方法である。
\begin{eqnarray}
	\frac{\partial}{\partial a}E(a,b) &=& 0\\
	\frac{\partial}{\partial b}E(a,b) &=& 0
\end{eqnarray}
から得られる方程式を、それぞれ$a$、$b$について解けば良く、それぞれの解を得るための方程式は次の2つを用いることになる。\cite{a}
\begin{equation}
	a = \frac{\sum_{n=1}^{n}(x_i -\bar{x})(y_i-\bar{y})}{\sum_{n=1}^{n}(x_i-\bar{x})^2}
	\label{eq:saisyou1}
\end{equation}
\begin{equation}
	b = \bar{y}-\frac{\sum_{n=1}^{n}(x_i -\bar{x})(y_i-\bar{y})}{\sum_{n=1}^{n}(x_i-\bar{x})^2} \bar{x}
	\label{eq:saisyou2}
\end{equation}

\section{方法}
\subsection{使用器具}
今回の実験で使用した器具を表\ref{tab:tools}に示す。
\begin{table}[hbtp]
	\caption{使用した実験器具}
	\label{tab:tools}
	\centering
	\begin{tabular}{|c|c|c|c|}
		\hline
		製品名 & 製造元 & 型番 & シリアル番号\\
		\hline \hline
		ノートパソコン & iiyama &NK50SZ& NKNK50SZ0000K0099\\
		\hline
		MyRIO & ANATEL &TiFi-BLE& 30877F9\\
		\hline
		MXP Breadboard & DIGILENT &MXP Breadboard& D537697\\
		\hline
		LabVIEW & NATIONAL INSTRUMENTS &LabVIEW2019& ver:19.0.1f3\\
		\hline
	\end{tabular}
	
\end{table}

\subsection{実験手順}
	\subsubsection{課題実験1}
		\begin{enumerate}[a.]
			\item ブレッドボートを図\ref{fig:circuit1}のように組み立てMyRIOのAポートに接続した。
			\item 図\ref{fig:circuit1}のAI0の端子を、GND, 3.3V, 5Vにつなぎ変え計測した。
			\item 出力した結果をもとにエクセルを使用し、平均値と標準偏差を求めた。
		\end{enumerate}

	\subsubsection{課題実験2}
		\begin{enumerate}[a.]
			\item ブレッドボートを図\ref{fig:circuit2}のように組み立てMyRIOのAポートに接続した。
			\item AI0へ0Vから5Vまで0.5V刻みでAO0から出力した。
			\item AO0の値、AI0の値、AO0からAI0を引いた値、通し番号を出力しExcelにまとめた。
			\item まとめたExcelから二乗平均平方誤差を求めた。
		\end{enumerate}

		\begin{figure}[H]
			\begin{tabular}{cc}
				\begin{minipage}[c]{0.5\linewidth}
				\centering
				\includegraphics[keepaspectratio,scale=1.2]{circuit1.pdf}
				\caption{課題実験1の回路図}
				\label{fig:circuit1}
				
			\end{minipage}&

			\begin{minipage}[c]{0.5\linewidth}
				\centering
				\includegraphics[keepaspectratio,scale=1.2]{circuit2.pdf}
				\caption{課題実験2の回路図}
				\label{fig:circuit2}
				
			\end{minipage}

			\end{tabular}

		\end{figure}

	\subsubsection{素子の電圧電流特性}
	各種実験は図\ref{fig:circuit3}の回路を使用し、測定対象素子の部分を変更しながら行った。
		\begin{enumerate}[1.]
			\item 固定抵抗の電圧電流特性
				\begin{enumerate}[a.]
					\item 観測対象素子を固定抵抗$1k[\Omega]$ とすた。
					\item $R_0$の抵抗値を$100[\Omega], 1k[\Omega], 10k[\Omega], 100k[\Omega]$と変化させた。
					\item 0.25[V]ずつ0~5[V]まで電圧を変化させ続け、その結果をもとに各抵抗における散布図を作成した。
				\end{enumerate}
			\item 可変抵抗の電圧電流特性
				\begin{enumerate}[a.]
					\item $R_0を1k[\Omega]$とし、図\ref{fig:potention}を参考に可変抵抗の端子間(1-2間、 2-3間, 1-3間 )を繋ぎ変えて測定した。
					\item また、各端子間の可変抵抗のつまみの位置をA, B, C にして計測した。
					\item 0.25[V]ずつ0~5[V]まで電圧を変化させ続けた。
					\item 各測定の結果をもとに、散布図を作成した。
				\end{enumerate}
				\begin{figure}[h]
					\centering
					\includegraphics[keepaspectratio,scale=1.2]{potention_paramerter.pdf}
					\caption{可変抵抗の詳細}
					\label{fig:potention}
				\end{figure}
			\item CdSセンサーの電圧電流特性
				\begin{enumerate}[a.]
					\item $R_0を1k[\Omega]$とし、素子の周りの環境が明るいところ、暗いところ、2箇所で測定をした。
					\item また、0.25[V]ずつ0~5[V]まで電圧を変化させ続けた。
					\item 測定結果を散布図にまとめた。 
				\end{enumerate}
			\item 力センサーの電圧電流特性
				\begin{enumerate}[a.]
					\item $R_0を1k[\Omega]$とし、素子を強く押した時、通常状態の2パターンで測定をした。
					\item また、0.25[V]ずつ0~5[V]まで電圧を変化させ続けた。
					\item 測定結果を散布図にまとめた。 
				\end{enumerate}
			\item 発光ダイオードの電圧電流特性 
				\begin{enumerate}[a.]
					\item $R_0を1k[\Omega]$とし、色の違う4色の発光ダイオードを使用し計測した。
					\item また、0.25[V]ずつ0~5[V]まで電圧を変化させ続けた。
					\item 測定結果を散布図にまとめた。 
				\end{enumerate}

				\begin{figure}[h]
					\centering
					\includegraphics[keepaspectratio,scale=1.2]{ex2_circuit1.pdf}
					\caption{素子の電圧電流特性計測用回路}
					\label{fig:circuit3}
				\end{figure}
		\end{enumerate}
		





\section{結果}
	\subsection{課題実験1の実験結果}
		プログラム\ref{program1}は課題実験1のプログラムであり結果\ref{fig:kekka1}はその出力結果である。
		表\ref{tab:0}は出力電圧が0[V]の時、表\ref{tab:3.3}は出力電圧が3.3[V]の時、表\ref{tab:5}は出力電圧が5[V]の時の出力結果の表である。
		また、下の表\ref{tab:hensa}は各電圧ごとの平均値と標準偏差をまとめたものである。
		
				

		\begin{table}[H]
			\caption{課題実験1の各電圧ごとの標準偏差}
			\centering
			\label{tab:hensa}
			\begin{tabular}{|c|c|c|}
				\hline
				電圧[V]	&	平均値[V]	&	標準誤差[V]\\
				\hline \hline
				0[V]		&0.007202&	1.68614\\
				\hline
				3.3[V]	&3.26400127&	1.33956\\
				\hline
				5[V]		&4.98062673&	1.68614\\
				\hline
			\end{tabular}
		\end{table}



		\begin{table}[H]
			\caption{0[V]の時の出力結果}
			\centering
			\label{tab:0}
			\begin{tabular}{|c|c|c|c|c|c|c|c|c|}
				\hline
			測定回数&測定電圧[V]&測定回数&測定電圧[V]&測定回数&測定電圧[V]&測定回数&測定電圧[V]\\
			\hline
			0  & 0.007324 & 25 & 0.007324 & 50 & 0.007324 & 75 & 0.007324 \\
			1  & 0.007324 & 26 & 0.007324 & 51 & 0.007324 & 76 & 0.007324 \\
			2  & 0.007324 & 27 & 0.007324 & 52 & 0.007324 & 77 & 0.007324 \\
			3  & 0.007324 & 28 & 0.007324 & 53 & 0.007324 & 78 & 0.007324 \\
			4  & 0.007324 & 29 & 0.007324 & 54 & 0.007324 & 79 & 0.007324 \\
			5  & 0.007324 & 30 & 0.007324 & 55 & 0.007324 & 80 & 0.007324 \\
			6  & 0.007324 & 31 & 0.007324 & 56 & 0.007324 & 81 & 0.007324 \\
			7  & 0.007324 & 32 & 0.007324 & 57 & 0.007324 & 82 & 0.007324 \\
			8  & 0.007324 & 33 & 0.007324 & 58 & 0.007324 & 83 & 0.007324 \\
			9  & 0.007324 & 34 & 0.007324 & 59 & 0.007324 & 84 & 0.007324 \\
			10 & 0.007324 & 35 & 0.006104 & 60 & 0.007324 & 85 & 0.006104 \\
			11 & 0.007324 & 36 & 0.006104 & 61 & 0.007324 & 86 & 0.006104 \\
			12 & 0.007324 & 37 & 0.006104 & 62 & 0.007324 & 87 & 0.007324 \\
			13 & 0.007324 & 38 & 0.006104 & 63 & 0.007324 & 88 & 0.007324 \\
			14 & 0.007324 & 39 & 0.007324 & 64 & 0.007324 & 89 & 0.006104 \\
			15 & 0.007324 & 40 & 0.007324 & 65 & 0.007324 & 90 & 0.006104 \\
			16 & 0.007324 & 41 & 0.007324 & 66 & 0.007324 & 91 & 0.007324 \\
			17 & 0.007324 & 42 & 0.007324 & 67 & 0.007324 & 92 & 0.007324 \\
			18 & 0.007324 & 43 & 0.007324 & 68 & 0.007324 & 93 & 0.007324 \\
			19 & 0.007324 & 44 & 0.007324 & 69 & 0.007324 & 94 & 0.007324 \\
			20 & 0.007324 & 45 & 0.007324 & 70 & 0.007324 & 95 & 0.007324 \\
			21 & 0.007324 & 46 & 0.007324 & 71 & 0.007324 & 96 & 0.007324 \\
			22 & 0.007324 & 47 & 0.007324 & 72 & 0.007324 & 97 & 0.006104 \\
			23 & 0.007324 & 48 & 0.007324 & 73 & 0.007324 & 98 & 0.006104 \\
			24 & 0.007324 & 49 & 0.007324 & 74 & 0.007324 & 99 & 0.007324\\
			\hline
			\end{tabular}
			\end{table}

		\begin{table}[H]
			\centering
			\caption{3.3[V]の時の出力結果}
			\label{tab:3.3}
			\begin{tabular}{|c|c|c|c|c|c|c|c|}
				\hline
			測定回数&測定電圧[V]&測定回数&測定電圧[V]&測定回数&測定電圧[V]&測定回数&測定電圧[V]\\
			\hline
			0  & 3.26416  & 25 & 3.26416  & 50 & 3.26416 & 75 & 3.262939 \\
			1  & 3.26416  & 26 & 3.26416  & 51 & 3.26416 & 76 & 3.262939 \\
			2  & 3.26416  & 27 & 3.26416  & 52 & 3.26416 & 77 & 3.26416  \\
			3  & 3.26416  & 28 & 3.262939 & 53 & 3.26416 & 78 & 3.26416  \\
			4  & 3.26416  & 29 & 3.262939 & 54 & 3.26416 & 79 & 3.26416  \\
			5  & 3.262939 & 30 & 3.26416  & 55 & 3.26416 & 80 & 3.26416  \\
			6  & 3.262939 & 31 & 3.26416  & 56 & 3.26416 & 81 & 3.26416  \\
			7  & 3.26416  & 32 & 3.26416  & 57 & 3.26416 & 82 & 3.262939 \\
			8  & 3.26416  & 33 & 3.26416  & 58 & 3.26416 & 83 & 3.262939 \\
			9  & 3.26416  & 34 & 3.26416  & 59 & 3.26416 & 84 & 3.26416  \\
			10 & 3.26416  & 35 & 3.26416  & 60 & 3.26416 & 85 & 3.26416  \\
			11 & 3.26416  & 36 & 3.26416  & 61 & 3.26416 & 86 & 3.26416  \\
			12 & 3.26416  & 37 & 3.26416  & 62 & 3.26416 & 87 & 3.26416  \\
			13 & 3.26416  & 38 & 3.26416  & 63 & 3.26416 & 88 & 3.26416  \\
			14 & 3.26416  & 39 & 3.26416  & 64 & 3.26416 & 89 & 3.26416  \\
			15 & 3.26416  & 40 & 3.26416  & 65 & 3.26416 & 90 & 3.26416  \\
			16 & 3.26416  & 41 & 3.26416  & 66 & 3.26416 & 91 & 3.26416  \\
			17 & 3.26416  & 42 & 3.26416  & 67 & 3.26416 & 92 & 3.262939 \\
			18 & 3.26416  & 43 & 3.26416  & 68 & 3.26416 & 93 & 3.262939 \\
			19 & 3.26416  & 44 & 3.26416  & 69 & 3.26416 & 94 & 3.26416  \\
			20 & 3.26416  & 45 & 3.26416  & 70 & 3.26416 & 95 & 3.26416  \\
			21 & 3.26416  & 46 & 3.26416  & 71 & 3.26416 & 96 & 3.262939 \\
			22 & 3.26416  & 47 & 3.26416  & 72 & 3.26416 & 97 & 3.262939 \\
			23 & 3.26416  & 48 & 3.262939 & 73 & 3.26416 & 98 & 3.26416  \\
			24 & 3.26416  & 49 & 3.26416  & 74 & 3.26416 & 99 & 3.26416 \\
			\hline
			\end{tabular}
			\end{table}

			\begin{table}[H]
				\centering
				\caption{5[V]の時の出力結果}
				\label{tab:5}
				\begin{tabular}{|c|c|c|c|c|c|c|c|c|}
					\hline
					測定回数&{測定電圧[V]}&測定回数&{測定電圧[V]}&測定回数&{測定電圧[V]}&測定回数&{測定電圧[V]}\\
					\hline
				0  & 4.981689 & 25 & 4.980468 & 50 & 4.980468 & 75 & 4.980468 \\
				1  & 4.980468 & 26 & 4.980468 & 51 & 4.980468 & 76 & 4.980468 \\
				2  & 4.980468 & 27 & 4.980468 & 52 & 4.980468 & 77 & 4.980468 \\
				3  & 4.981689 & 28 & 4.980468 & 53 & 4.980468 & 78 & 4.980468 \\
				4  & 4.980468 & 29 & 4.980468 & 54 & 4.980468 & 79 & 4.980468 \\
				5  & 4.980468 & 30 & 4.980468 & 55 & 4.980468 & 80 & 4.980468 \\
				6  & 4.980468 & 31 & 4.980468 & 56 & 4.980468 & 81 & 4.980468 \\
				7  & 4.980468 & 32 & 4.980468 & 57 & 4.980468 & 82 & 4.980468 \\
				8  & 4.980468 & 33 & 4.980468 & 58 & 4.980468 & 83 & 4.980468 \\
				9  & 4.980468 & 34 & 4.980468 & 59 & 4.981689 & 84 & 4.980468 \\
				10 & 4.981689 & 35 & 4.980468 & 60 & 4.981689 & 85 & 4.980468 \\
				11 & 4.981689 & 36 & 4.980468 & 61 & 4.980468 & 86 & 4.980468 \\
				12 & 4.980468 & 37 & 4.980468 & 62 & 4.980468 & 87 & 4.980468 \\
				13 & 4.980468 & 38 & 4.981689 & 63 & 4.980468 & 88 & 4.980468 \\
				14 & 4.981689 & 39 & 4.981689 & 64 & 4.980468 & 89 & 4.980468 \\
				15 & 4.981689 & 40 & 4.980468 & 65 & 4.980468 & 90 & 4.980468 \\
				16 & 4.981689 & 41 & 4.980468 & 66 & 4.980468 & 91 & 4.980468 \\
				17 & 4.980468 & 42 & 4.980468 & 67 & 4.980468 & 92 & 4.980468 \\
				18 & 4.980468 & 43 & 4.980468 & 68 & 4.980468 & 93 & 4.980468 \\
				19 & 4.980468 & 44 & 4.980468 & 69 & 4.980468 & 94 & 4.980468 \\
				20 & 4.980468 & 45 & 4.980468 & 70 & 4.980468 & 95 & 4.980468 \\
				21 & 4.980468 & 46 & 4.980468 & 71 & 4.981689 & 96 & 4.980468 \\
				22 & 4.980468 & 47 & 4.980468 & 72 & 4.981689 & 97 & 4.980468 \\
				23 & 4.980468 & 48 & 4.980468 & 73 & 4.980468 & 98 & 4.980468 \\
				24 & 4.980468 & 49 & 4.980468 & 74 & 4.980468 & 99 & 4.980468\\
				\hline
				\end{tabular}
			\end{table}

				\begin{table}[H]
					\centering
					\caption{課題実験2の出力結果}
					\label{tab:task2}
					\begin{tabular}{|c|c|c|c|c|}
					\hline
					計測回数[回]&電圧[V]&計測電圧[V]&誤差[V]\\
					\hline
					0  & 0   & 0.007324 & -0.00732 \\
					1  & 0.5 & 0.499268 & 0.000732 \\
					2  & 1   & 0.997314 & 0.002686 \\
					3  & 1.5 & 1.496582 & 0.003418 \\
					4  & 2   & 1.995849 & 0.004151 \\
					5  & 2.5 & 2.496338 & 0.003662 \\
					6  & 3   & 2.996826 & 0.003174 \\
					7  & 3.5 & 3.497314 & 0.002686 \\
					8  & 4   & 3.997802 & 0.002198 \\
					9  & 4.5 & 4.498291 & 0.001709 \\
					10 & 5   & 4.997558 & 0.002442\\
					\hline
					\end{tabular}
					\end{table}

	\subsection{課題実験2の実験結果}
		プログラム\ref{program2}は課題実験1のプログラムであり結果\ref{fig:kekka2}はその出力結果である。
		表\ref{tab:task2}は課題実験2の出力結果である。
		またこの結果より二乗平均平方誤差は1.49994[V]となった


		\begin{figure}[H]
			\begin{tabular}{cc}
				\begin{minipage}[c]{0.5\linewidth}
					\centering
					\includegraphics[keepaspectratio,scale=0.5]{circuit1.png}
					\useMycounter[\label{program1}]課題実験1
	
				\end{minipage}&
	
				\begin{minipage}[c]{0.5\linewidth}
					\centering
					\includegraphics[keepaspectratio,scale=0.5]{circuit2.png}
					\useMycounter[\label{program2}]課題実験2
					
				\end{minipage}\\\\\\

				

				\begin{minipage}[c]{0.5\linewidth}
					\centering
					\includegraphics[keepaspectratio,scale=0.4]{kekka1.png}
					\caption{課題実験1の出力結果}
					\label{fig:kekka1}
					
				\end{minipage}&
	
				\begin{minipage}[c]{0.5\linewidth}
					\centering
					\includegraphics[keepaspectratio,scale=0.4]{kekka2.png}
					\caption{課題実験2の出力結果}
					\label{fig:kekka2}
					
				\end{minipage}
				\end{tabular}
			\end{figure}

	\subsection{素子の電圧電流特性の実験結果}
	プログラム\ref{program3}は素子の電圧電流特性を計測するプログラムであり、この実験における結果は図\ref{fig:result}のように出力された。
	\begin{figure}[H]
		\centering
		\includegraphics[keepaspectratio,scale=0.4]{program.png}
		\useMycounter[\label{program3}]素子の電圧電流特性の実験におけるプログラム
	\end{figure}
	\begin{figure}[H]
		\centering
		\includegraphics[keepaspectratio,scale=0.4]{result.png}
		\caption{素子の電圧電流特性の実験における出力結果}
		\label{fig:result}
	\end{figure}
	\subsubsection{固定抵抗の電圧電流特性}
	図\ref{fig:graph1}は測定素子が固定抵抗の際の実験結果から電圧電流特性の散布図にしたものである。
		\begin{figure}[H]
			\centering
			\includegraphics[keepaspectratio,scale=1.1]{m20060/experoment/eex31.pdf}
			\caption{固定抵抗の電圧電流特性}
			\label{fig:graph1}
		\end{figure}
	\subsubsection{可変抵抗の電圧電流特性}
	図\ref{fig:graph2-12}は測定素子が可変抵抗の際かつ、端子間1−2の時の実験結果から電圧電流特性の散布図にしたものである。
	また、図\ref{fig:graph2-23}は端子間2−3の散布図であり、図\ref{fig:graph2-13}は端子間1−3の散布図である。
		\begin{figure}[H]
			\centering
			\includegraphics[keepaspectratio,scale=1.1]{m20060/experoment/ee32_12.pdf}
			\caption{可変抵抗(1−2端子間)の電圧電流特性}
			\label{fig:graph2-12}
		\end{figure}
		\begin{figure}[H]
			\centering
			\includegraphics[keepaspectratio,scale=1.1]{m20060/experoment/ee32_23.pdf}
			\caption{可変抵抗(2−3端子間)の電圧電流特性}
			\label{fig:graph2-23}
		\end{figure}
		\begin{figure}[H]
			\centering
			\includegraphics[keepaspectratio,scale=1.1]{m20060/experoment/ee32_13.pdf}
			\caption{可変抵抗(1−3端子間)の電圧電流特性}
			\label{fig:graph2-13}
		\end{figure}

	\subsubsection{CdSセンサーの電圧電流特性}
	図\ref{fig:graph3}は測定素子がCdSセンサーの際の実験結果から電圧電流特性の散布図にしたものである。
	\begin{figure}[H]
		\centering
		\includegraphics[keepaspectratio,scale=1.1]{m20060/experoment/ee33.pdf}
		\caption{CdSセンサーの電圧電流特性}
		\label{fig:graph3}
	\end{figure}
	\subsubsection{力センサーの電圧電流特性}
	図\ref{fig:graph4}は測定素子が力センサーの際の実験結果から電圧電流特性の散布図にしたものである。
	\begin{figure}[H]
		\centering
		\includegraphics[keepaspectratio,scale=1.1]{m20060/experoment/ee4.pdf}
		\caption{力センサーの電圧電流特性}
		\label{fig:graph4}
	\end{figure}
	\subsubsection{発光ダイオードの電圧電流特性}
	図\ref{fig:graph5}は測定素子が発光ダイオードの際の実験結果から電圧電流特性の散布図にしたものである。
	\begin{figure}[H]
		\centering
		\includegraphics[keepaspectratio,scale=1.1]{m20060/experoment/ex5.pdf}
		\caption{発光ダイオードの電圧電流特性}
		\label{fig:graph5}
	\end{figure}

\section{考察}
	\subsection{課題実験1の考察}
	この実験における理想の平均値と標準偏差はそれぞれ5.0[V], 3.0[V], 0[V]である。
	なぜならAO0から5.0[V], 3.3[V], 0.0[V]を出力しているためAI0からの入力もその値になることが理想である。
	また、標準偏差も理想は出力値と計測値が同じはずなので0になることが理想であると考察する。

	\subsection{課題実験2の実験結果の考察}
	図\ref{fig:kinji}は出力電圧表示値と計測電圧表示値の関係をグラフにし、近似曲線求めたものである。
	またこの近似曲線における傾きの理想の値は1であり、切片は0であると考える。
	実際に傾きを式\ref{katamuki}から求めると$a=0.999289673$となった。
	また、切片は表\ref{tab:task2}より$0.007324$であった。
	このように理想の値と実際の値で差が生じた原因としてMyRIOのポートにあるプルアップ抵抗によって生じたのだと考える。


	\begin{equation}
		a=\frac{\Delta y}{\Delta x}
		\label{katamuki}
	\end{equation}

	
	\begin{figure}[H]
		\centering
		\includegraphics[keepaspectratio,scale=0.25]{m20060/kinji.png}
		\caption{出力電圧と計測電圧の散布図と近似曲線}
		\label{fig:kinji}
	\end{figure}

	\subsection{素子の電圧電流特性の実験結果の考察}
	\subsubsection{固定抵抗の考察}
	$R_0$が$100[\Omega], 1k[\Omega], 10k[\Omega], 100k[\Omega]$の時それぞれ傾きの平均は
	$0.0010056, 0.0010004, 0.0010075, 0.0009324$となった。この値の逆数をとると、
	$995[\Omega], 1000[\Omega], 993[\Omega], 1147[\Omega]$となり公称値$1000[\Omega]$と比較するとほぼ$1000[\Omega]$ということがわかる。
	また、$R_0$の変化が算出した抵抗値に与える影響については、抵抗値が小さすぎても、大きすぎても、誤差が大きくなってしまう点だと考察する。

	\subsubsection{可変抵抗の電圧電流特性の実験結果の考察}
	各端子間のつまみ位置による電圧電流特性の変化の仕方から、図\ref{fig:potention-inside}のような内部構造を持っていると推測できる。
	この図において、端子2から繋がっている矢印は可変抵抗のつまみを示している。
	また、各A, B, Cは実験で用いたつまみの位置を示している。
	\begin{figure}[H]
		\centering
		\includegraphics[keepaspectratio,scale=1.1]{potention_paramerter_inside.pdf}
		\caption{電圧電流特性から推測した可変抵抗の内部構造}
		\label{fig:potention-inside}
	\end{figure}

	\subsubsection{CdSセンサーの電圧電流特性の実験結果の考察}
	CdSセンサーの抵抗値は光強度が増加することで減少すると考察する。\\
	“CdS”とは、硫化カドミウム(CdS)を主成分とする光導電素子の一種で、光の当たる量によって抵抗値が変化します。CdSセルの受光部に当たる光の量が多ければ、抵抗値が小さくなります。\cite{x}\\
	このことから実験結果は正しいことが確認できる。

	\subsubsection{力センサーの電圧電流特性の実験結果の考察}
	力センサーの抵抗値は力が増加すると減少すると考察する。\\
	圧電式力センサ(水晶圧電式センサ)は、2組の水晶板と、その間の電極箔で構成されます。水晶は特定の結晶軸方向に力が加わると、その力に比例した電荷を発生するため、
	この圧電効果を利用してひずみを計測できます。得られる信号は電荷であるため、チャージアンプを用いて値を電圧信号に変換した上で計測を行います。\cite{b}\\
	このことから実験結果は正しいことが確認できる。

	\subsubsection{発光ダイオードの電圧電流特性の実験結果の考察}
	発光ダイオードにおいて色の変化とは光の波長に関係する。その光の波長は発光ダイオード内のp-n接合でのホールと電子
	の結合において電子がエネルギーの高い伝導帯からエネルギーの低い価電子帯に落ちるとき、このエネルギーの差が大きいほど波長が短い光が放出される。
	よってこのエネルギーの差によって光の色が異なる。図\ref{fig:graph5}から、各色ごとの傾きには大きな特徴などなく一定だが、
	立ち上がりに関して、赤系と青系の色では赤系の色の方が立ち上がり電圧が低い。
	また、赤系の色の方が発光ダイオード自体の抵抗も小さいということが考察できる。


\section{結論}
\begin{itemize}
	\item LabVIEWとMyRIOを使用して、素子の電圧電流特性について自動計測の方法を習得ことができた。
	\item 測定データから近似直線式の傾き、切片を求める計算方法を習得することができた。
\end{itemize}

\begin{thebibliography}{9}%参考文献数が10以上の場合は9を99に変更
	%\bibitem{xxx}の引用を本文中で行うには\cite{xxx}と記述。
	\bibitem{a} 阿部 武雄,村山 実, 電気・電子計測、森北出版,2022年2月21日
	\bibitem{x} セミナーeラーニングアイアール技術者教育研究所,"3分でわかる技術の超キホン CdSセルとは?原理と電子回路での使い方を解説!",技術のキホン,2021.04.22,https://engineer-education.com/cds-cell/,(参照2022.05.30)
	\bibitem{b} HBK,"力センサは、圧電式(ピエゾ式)?それともひずみゲージ式?",計測技術ライブラリ,記載なし,https://www.hbm.com/jp/3719/piezoelectric-or-strain-gauge-based-force-transducers,(参照2022.05.30)

\end{thebibliography}


\end{document}